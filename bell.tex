\documentclass[headsepline,titlepage,ngerman,twoside,12pt]{report}
\usepackage[utf8]{inputenc}
\usepackage[T1]{fontenc}
\usepackage[ngerman]{babel}
\usepackage[a4paper,top=4cm,bottom=3cm,left=4cm,right=3cm]{geometry}
\usepackage[pdftex]{graphicx}
\usepackage{setspace}
\usepackage{csquotes}
\usepackage{tabularx}
\usepackage{xcolor}
\usepackage{booktabs}
\usepackage{amsmath}
\usepackage{aurl}
\usepackage{microtype}
\usepackage{natbib}
\usepackage{hyperref}
\usepackage{acronym}
\usepackage{cleveref}
\PassOptionsToPackage{pdfborder={0 0 0}}{hyperref} %Für finale gedruckte Ausgabe, ohne hervorgehobene Links
%\usepackage{hypernat}
\daurl{ob}{http://www.snik.eu/ontology/ob/}
\daurl{bb}{http://www.snik.eu/ontology/bb/}
\daurl{ciox}{http://www.snik.eu/ontology/ciox/}
\daurl{he}{http://www.snik.eu/ontology/he/}
\daurl{it4it}{http://www.snik.eu/ontology/it4it/}
\daurl{meta}{http://www.snik.eu/ontology/meta/}
\author{Max Niclas Wächtler}
%\newcommand\todo[1]{\textcolor{teal}{#1}}%show comments for students, comment out for finished work or remove the todo statements and their content
\newcommand\todo[1]{}%hide comments for students, uncomment for the finished work 
%remove "Kapitel X" headers *****
\begin{document}
\begin{acronym}
\acro{SPARQL}{SPARQL Protocol and RDF Query Language}
\acro{W3C}{World Wide Web Consortium}
\acro{KI}{Künstliche Intelligenz}
\acro{RDF}{Resource Description Framework}
\acro{HTML}{Hypertext Markup Language}
\acro{XML}{Extensible Markup Language}
\acro{URIs}{Universal Resource Identifier}
\end{acronym}
\allowdisplaybreaks%Mathe darf auch umbrechen

\onehalfspace

\begin{titlepage}
\thispagestyle{empty}
\begin{center}

{\large\bf UNIVERSITÄT LEIPZIG\\[1mm]}
Institut für Medizinische Informatik, Statistik und Epidemiologie (IMISE)

\vspace*{4cm}

{\Huge\textbf{Automatische Erstellung von Quizfragen aus einer Ontologie von Krankenhausinformationssystemen}\\}
\vspace{0.5cm}
{\large Besondere Lernleistung}\\
\vspace{2cm}
\todo{15--20 Seiten. Zwei Möglichkeiten:\\
(1) Vergleich, Zusammenfassung und \textbf{eigene} Bewertung zweier wissenschaftlicher Paper.\\
(2) Nutzenbewertung eines wissenschaftlichen Papers hinsichtlich einer konkreten Forschungssituation
}

\vspace*{4cm}

\begin{tabularx}{\textwidth}{lr}
Leipzig, <Monat> <Jahr>		&vorgelegt von:\\
\\
				&\makeatletter Max Niclas Wächtler \makeatother\\
				&geb. am: 17.05.2003\\
\\
				&Betreuer:\\
				&Konrad Höffner\\
\end{tabularx}
\vspace{1cm}

\end{center}
\end{titlepage}
\newpage
\begin{abstract}
Zusammenfassung hier schreiben
\end{abstract}
\tableofcontents
\newpage

\section*{Schlagwörter}
\todo{
Verschlagworten Sie Ihre Seminararbeit.
Mindestens 3 Schlagworte.
Wenn möglich nutzen Sie dazu Begriffe aus der SNIK-Ontologie, inbesondere bei Arbeiten aus dem Bereich Management von Informationssystemen.
Sie finden die URLs über die Suche auf SNIK Graph unter \url{http://www.snik.eu/graph}.
Achten Sie auch in der Seminararbeit auf die Verwendung dieser Schlagworte.
Eine Anleitung dazu finden Sie in der Auftaktveranstaltung.
}

{
\centering
\begin{tabularx}{\textwidth}{XX}
\toprule
Schlagwort		&SNIK-URL\\
\midrule
Informationsmanagement	&\aurl{ob}{InformationManagement}\\
\bottomrule
\end{tabularx}
}

\section*{Begriffs- und Abkürzungsverzeichnis}
\begin{tabularx}{\textwidth}{lX}
\toprule
\textrm{Begriff}			&\textrm{Erklärung des Begriffs}\\
\midrule
ARIS					&Architektur integrierter Informationssysteme\\
$\ldots$				&$\ldots$\\
$\ldots$				&$\ldots$\\
$\ldots$				&$\ldots$\\
$\ldots$				&$\ldots$\\
$\ldots$				&$\ldots$\\
$\ldots$				&$\ldots$\\
$\ldots$				&$\ldots$\\
$\ldots$				&$\ldots$\\
UKL					&Universitätsklinikum Leipzig\\
3LGM$^2$				&Drei-Ebenen-Metamodell\\
\bottomrule
\end{tabularx}

\chapter{Einleitung}
\section{Gegenstand und Motivation \todo{(aus Sicht der Autoren der Paper)}}
\subsection{Gegenstand}
\todo{
\begin{itemize}
\item In welcher Welt/Domäne oder welchem Arbeitsbereich/-gebiet bewegen wir uns im Rahmen der Seminararbeit/der ausgewählten Papers?
\item Worum geht es eigentlich?
\end{itemize}
Aus den Papers bzw. dem Antrag des Forschungsprojektes entnehmen.
}

\subsection{Problematik}
\todo{
\begin{itemize}
\item Worin bestehen die Probleme?
\item Warum ist die geschilderte vorliegende Situation problematisch?
\item Für wen ist sie problematisch?
\end{itemize}
Aus den Papers bzw. dem Antrag des Forschungsprojektes entnehmen.\\
}
\todo{
Hier ist auf die generelle Problemlage einzugehen, die dem Aufsatz im Wissenschaftsfeld zu Grunde liegt.\\
Bsp.: Daten aus der Krankenversorgung stehen aus rechtlichen und technischen Gründen nicht für die Forschung zur Verfügung, sodass klinische Forscher eigene Datenerhebungen planen müssen.
}
\subsection{Motivation}
\todo{
\begin{itemize}
\item Warum lohnt es sich, die genannten Probleme zu lösen?
\item Wer wird welchen Nutzen von dieser Abschlussarbeit haben?
\item Warum sind die Papers wichtig?
\end{itemize}
Aus den Papers bzw. dem Antrag des Forschungsprojektes entnehmen.
}
\todo{
Worin liegt der zusätzliche Nutzen, die beiden Ansätze zu vergleichen? Worin liegt der Nutzen, den Ansatz des Papers auf die Problematik des Forschungsprojektes anzuwenden?\\
Bsp.: Gelänge es, Daten aus der Krankenversorgung vollständig zu anonymisieren, könnten sie für beliebige Forschungsvorhaben genutzt werden.
}

\section{Zielsetzung}
\todo{
Beschreiben Sie kurz die Zielsetzung ihrer Seminararbeit.
Beachten Sie dazu die Beschreibung des Betreuers zu Ihrem Seminarthema.\\
Beispiel:\\
Es soll herausgearbeitet werden, ob einer der in den Papers vorgestellten Lösungsansätze im aktuellen Forschungsprojekt angewendet werden kann: „Untersuchung der Eignung des Anonymisierungskonzept der statistischen Landesämter für medizinische Daten unter besonderer Betrachtung der Risiken des Zusatzwissens behandelnder Ärzte.“
Es soll eine Übersicht erarbeitet werden, die es ermöglicht, in einem konkreten Anwendungsfall den dort geeigneten Lösungsansatz auszuwählen.
}

\section{Vorgehensweise}
\todo{
Stellen Sie Ihre Methode bzw. Herangehensweise dar.
Was ist Ihre konkrete Aufgabenstellung im Rahmen der Seminararbeit? Das könnte z.B. sein:
\begin{itemize}
\item Vorstellen verschiedener Lösungsansätze, Vergleich und Bewertung der Lösungsansätze, Literaturrecherche nach weiteren vergleichbaren Papers.
\item Bewertung eines Lösungsansatzes hinsichtlich einer konkreten Forschungssituation. Literaturrecherche nach weiteren vergleichbaren Papers.
\end{itemize}
}
\todo{
Beschreiben Sie Ihre Vorgehensweise in einzelnen Schritten und stellen Sie die Reihenfolge der Abarbeitung dar.
Auf diese Punkte sollten Sie in der Diskussion wieder eingehen.
}
\todo{
\begin{enumerate}
\item Schritt 1 
\item Schritt 2
\end{enumerate}
}
\todo{
Vermeiden Sie Formulierungen wie \enquote{Es ist nicht bekannt, ob...} oder \enquote{Es existiert kein...}.
Solche Formulierungen kehren in der Regel einfach das bereits angedachte Lösungsmodell um und postulieren das Fehlen der angedachten Lösung einfach als Problem.
\item ... 
\item Schritt n
Das ist ähnlich, wie wenn es in der Werbung hieße \enquote{Wenn Sie das Problem haben, dass Ihnen Aspirin fehlt, dann kaufen Sie doch Aspirin.}
Sinnvoller ist diese Aussage: \enquote{Wenn Sie das Problem haben, dass Ihnen der Kopf weh tut, dann kaufen Sie doch Aspirin.}
Es ist also bei der Problembeschreibung erforderlich, sich in die Lage dessen zu versetzen, den man mit der angedachten Lösung beglücken möchte.
Sein Problem ist zu ermitteln und so zu formulieren, er/sie das Problem wiedererkennt und dadurch geneigt ist, sich für die Lösung des Problems zu interessieren.
}

%\todo{(2--3 Seiten)}
\chapter{Grundlagen }
\todo{
Erklären Sie alle fachlichen Grundlagen, die notwendig sind, um der Bearbeitung Ihres Seminar­themas folgen zu können.
Beachten Sie, dass die Darstellung für den durchschnittlichen Vorlesungs­hörer verständlich ist.
}
\section{RDF}
\ac{RDF} ist ein Standardmodell für den Datenaustausch im Web, welches in den 90er Jahren des letzten Jahrtausends seine Anfänge findet. Im Gegensatz zu \ac{HTML} und \ac{XML} ist es in der Lage, enthaltene Informationen zu kombinieren und weiter zu verarbeiten, anstatt diese nur korrekt anzuzeigen. Daher wird RDF auch oft als grundlegendes Darstellungsformat für die Entwicklung im Semantic Web angesehen.
Es erweitert die Verknüpfungsstruktur von diesem, um \ac{URIs} einzubinden.
Solche URIs werden im semantischen Netz dazu genutzt, Beziehungen zwischen Dingen beziehungsweise zwischen den beiden Enden der Verknüpfung herzustellen, um Dateien zu verknüpfen, verfügbar zu machen und für Anwendungen freizugeben. Dadurch können Daten zusammengeführt werden, die sich eigentlich im Schema grundlegend unterscheiden, ohne dass alle Datenkonsumenten geändert werden müssen. Diese Zusammenführung geschieht in einem Schichtenmodell: Aussagen werden als Tripel modelliert und bilden eine Aussage, die aus Subjekt, Prädikat und Objekt zusammengefügt werden kann. Man kann also einen Tripel mit einem einfachen Satz vergleichen: als Beispiel hier einmal der Teilsatz \textit{Google produziert Prozessoren}. Hier stellt \textit{Google} das Subjekt dar, \textit{produziert} ist das Prädikat und \textit{Prozessoren} sind das Objekt. Hat man eine Menge von Tripeln, bilden diese ein semantisches Netz, was man sich dann in dem Fall einfach tabellarisch vorstellen kann, siehe \cref{tab:rdfexample}.

\begin{table}
\begin{centering}
\begin{tabularx}{\textwidth}{XXX}
\toprule
\textrm{Subjekt}			&\textrm{Prädikat}			&\textrm{Objekt}\\
\midrule
Paint					&erstellt				&Grafiken\\
Rasenmäher				&mäht					&Rasen\\
Google					&produziert				&Prozessoren\\
\bottomrule
\end{tabularx}
\end{centering}
\caption{Beispiel für RDF-Tripel}
\label{tab:rdfexample}
\end{table}
\section{SPARQL}
\ac{SPARQL} ist ein recht neuer Standard für die Abfrage von in RDF spezialisierten Informationen sowie für die Darstellung der zugehörigen Resultate. Es basiert auf einfachen RDF-Anfragen in Form von Graphmustern, enthält jedoch auch erweiterte Funktionen für die Konstruktion komplexerer Anfragemuster, die Verwendung zusätzlich hinzufügbarer Filterbedingungen und für die Formatierung der schlussendlichen Ausgabe.
\section {Ontologien}
Ontologien sind (im informatischen Kontext) meist sprachlich gefasste und geordnete Darstellungen einer Menge von Begrifflichkeiten mit festen Beziehungen untereinander. Diese Beziehungen befassen sich stets auf einen bestimmten Gegenstandsbereich und werden dazu genutzt, Wissen in digitalisierter Form zwischen Anwendungen und Diensten auszutauschen. Dabei müssen bestimmte Interferenz- und Integritätsregeln eingehalten werden, also Regeln zu Schlussfolgerungen sowie zu der Gewährleistung ihrer Gültigkeit. Ontologien erfreuen sich seit der Idee des semantischen Webs einem stets wachsenden Bekanntheitsgrad, was dazu führt, dass sie als Teil der Wissensrepräsentation im Teilgebiet \ac{KI} einen großen Einfluss haben. Dabei beschreibt die Idee des Semantic Web eine Erweiterung des vorhandenen Netzes, um Daten zwischen Rechnern einfacher austausch- und verwertbar zu machen und sie somit besser zu explizieren, anstatt sie unkonstruiert stehen zu lassen. Zum Sinne dieser Realisierung dienen Standards zur Veröffentlichung und Nutzung maschinenlesbarer Daten, insbesondere RDF.
\chapter{Lösungsansatz/Lösungsansätze \todo{(bezogen auf Paper, 2--4 Seiten)}}
\todo{Kapitel 3 ist Vorschlag gedacht, der nicht wörtlich übernommen werden muss.
Der Lösungsansatz ist eine kurze Beschreibung der Arbeitshypothese sowie des Vorgehens zur Lösung der in der Einleitung beschriebenen Probleme.}

\section{Ergebnis/Lösungsansatz Paper n}
\section{Vergleich mit Projekt}

\chapter{Ergebnis \todo{der Seminararbeit, 1--2 Seiten, ihre Ergebnisse}}
\todo{
Im Ergebniskapitel soll beschrieben werden, inwiefern Sie Ihre in 1.2/1.3 aufgestellten Ziele bzw. Aufgaben im Rahmen der Seminararbeit erreicht wurden oder auch weshalb sie (teilweise) nicht erreicht werden konnten.
So soll es möglich sein, dass ein Leser von der Arbeit lediglich die Einleitung und die Zusammenfassung liest und doch die Ergebnisse der Arbeit erfassen kann.
Dieses Kapitel kann auch in mehrere Unterkapitel aufgeteilt werden, wenn das sinnvoll ist!
}

\chapter{Diskussion und Ausblick \todo{1--2 Seiten}}
\todo{
In der Diskussion wird das Ergebnis der Arbeit noch einmal kritisch bewertet.
Dies umfasst auch neue Probleme, die erst während der Bearbeitung erkannt wurden.
Die Eignung für ein konkretes Forschungsprojekt sollte hier diskutiert werden.
In der Diskussion kann auch die Kritik des Autors an dem stehen, was er in der Literatur zu dem zu bearbeitenden Thema hier und da gelesen hat.
Außerdem soll ein Ausblick gegeben werden, welche weiteren Fragestellungen noch bearbeitet werden sollten.
}

\renewcommand{\bibpreamble}{
\todo{
Zentrale Literatur, die den untersuchten Artikeln zugrunde liegt, können Sie im bibtex-Format in die seminar.bib einfügen und dann im Text zitieren, wodurch diese automatisch in das Literaturverzeichnis übernommen werden.
Sie können auch weitere referierte Veröffentlichungen (wissenschaftliche Zeitschriften (auch elektronisch), Bücher) in die Erarbeitung einbeziehen zitieren.
Sie können dafür Literaturdatenbanken wie Pubmed oder Google Scholar durchsuchen und dort direkt als bibtex-Eintrag herunterladen.
Bitte beachten Sie, dass die \href{https://www.openoffice.org/bibliographic/bibtex-defs.html}{bibtex-Einträge vollständig sind}, z.B. sind das bei Büchern Autor, Editor, Title, Kapitel oder Seiten, Herausgeber und Jahr..
Weitere Informationen zum Thema der Literaturrecherche finden sich in den \href{http://www.imise.uni-leipzig.de/Lehre/MedInf/Abschlussarbeiten/Literaturrecherche.jsp}{Hinweisen zur Literaturrecherche}.
\paragraph{Beispielzitierungen}
\citet{his} beschreiben ein Verfahren zur X von Y auf Basis von Z.
Alternativ: X von Y lässt sich auf Basis von Z ermitteln~\citep{his}.
}
}



\end{document}

