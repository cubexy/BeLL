\documentclass[headsepline,titlepage,ngerman,twoside,12pt]{report}
\usepackage[utf8]{inputenc}
\usepackage[T1]{fontenc}
\usepackage[ngerman]{babel}
\usepackage[a4paper,top=4cm,bottom=3cm,left=4cm,right=3cm]{geometry}
\usepackage[pdftex]{graphicx}
\usepackage{setspace}
\usepackage{csquotes}
\usepackage{tabularx}
\usepackage{xcolor}
\usepackage{listings}
\usepackage{booktabs}
\usepackage{amsmath}
\usepackage{comment}
\usepackage{aurl}
\usepackage{microtype}
\usepackage{natbib}
\usepackage{hyperref}
\usepackage{acronym}
\usepackage{cleveref}
\usepackage{epigraph}
\PassOptionsToPackage{pdfborder={0 0 0}}{hyperref} %Für finale gedruckte Ausgabe, ohne hervorgehobene Links
%\usepackage{hypernat}
\daurl{ob}{http://www.snik.eu/ontology/ob/}
\daurl{bb}{http://www.snik.eu/ontology/bb/}
\daurl{ciox}{http://www.snik.eu/ontology/ciox/}
\daurl{he}{http://www.snik.eu/ontology/he/}
\daurl{it4it}{http://www.snik.eu/ontology/it4it/}
\daurl{meta}{http://www.snik.eu/ontology/meta/}
\author{Max Niclas Wächtler}
\newcommand\idea[1]{\textcolor{red}{#1}}

\newcommand\todo[1]{TODO: \textcolor{teal}{#1}}
%show comments for students, comment out for finished work or remove the todo statements and their content
%\newcommand\todo[1]{}%hide comments for students, uncomment for the finished work 
%remove "Kapitel X" headers *****
\setlength{\parindent}{0em}

\begin{document}
\allowdisplaybreaks%Mathe darf auch umbrechen
\lstset{language=SQL,morekeywords={PREFIX,owl,rdf,rdfs,meta,FILTER,str,bb,ob,SAMPLE,STR,AS,REPLACE,LANGMATCHES,skos,LANG,LIMIT}}

\onehalfspace

\begin{titlepage}
\thispagestyle{empty}
\begin{center}

{\large\bf UNIVERSITÄT LEIPZIG\\[1mm]}
Institut für Medizinische Informatik, Statistik und Epidemiologie (IMISE)

\vspace*{4cm}

{\Huge\textbf{Automatische Erstellung von Quizfragen aus einer Ontologie von Krankenhausinformationssystemen}\\}
\vspace{0.5cm}
{\large Besondere Lernleistung}\\
\vspace{2cm}

\vspace*{4cm}

\begin{tabularx}{\textwidth}{Xr}
Leipzig, Dezember 2020		&vorgelegt von:\\
\\
				&Max Niclas Wächtler\\
				&geb. am: 17.05.2003\\
\\
				&Betreuer:\\
				&Konrad Höffner\\
\end{tabularx}
\vspace{1cm}

\end{center}
\end{titlepage}
\newpage
\begin{abstract}
%Kurzreferat von Januar
Krankenhausinformationssysteme sind Informationssysteme, die sich in medizinischen Umgebungen mit der Verarbeitung von Daten, Informationen und Wissen beschäftigen.
Informationssysteme bestehen aus Hardware, Software als auch Anwendern. Kompetenz in der Nutzung von Krankenhausinformationssystemen ist wichtig, um in Krankenhäusern beispielsweise die Zusammenarbeit mit Patienten effektiv und zeitsparend zu gestalten.
Jedoch müssen Anwender diese Kompetenz erst einmal durch Quellen wie Lehrbücher erwerben, welche jedoch verschiedene Sichten auf das Themengebiet aufweisen und keinen schnellen Überblick über bestimmte Themen bieten können. Um dieses Problem anzugehen, habe ich mir als Ziel dieser besonderen Lernleistung gesetzt, ein Programm zu entwickeln, welches diese Informationen aus einer Ontologie, also eine Menge von Begrifflichkeiten mit Beziehungen untereinander, über eine Schnittstelle entnimmt, diese nutzerfreundlich in Form von zufällig generierten Fragen aufarbeitet und dem Nutzer ausgibt. Diese sollen eine grammatikalisch korrekte Form aufweisen und nebst der richtigen Antwort auch falsche Antworten enthalten, die trotzdem logisch wirken sollen, anstatt ein Zufallsverfahren zur Auswahl zu nutzen.  
\end{abstract}
\tableofcontents
\newpage


\section*{Begriffs- und Abkürzungsverzeichnis}
\begin{acronym} [SPARQL]
\acro{SPARQL}{SPARQL Protocol and RDF Query Language}
\acro{W3C}{World Wide Web Consortium}
\acro{KI}{Künstliche Intelligenz}
\acro{RDF}{Resource Description Framework}
\acro{HTML}{Hypertext Markup Language}
\acro{XML}{Extensible Markup Language}
\acro{URIs}{Universal Resource Identifier}
\acro{SNIK}{Semantisches Netz des Informationsmanagements im Krankenhaus}
\acro{WWW}{World Wide Web}
\acro{W3C}{World Wide Web Consortium}
\acro{HTTP}{Hypertext Transfer Protocol}
\acro{MIG}{Management von Informationssystemen im Gesundheitswesen}
\end{acronym}

\chapter{Einleitung}
\section{Gegenstand und Motivation}
\subsection{Gegenstand}
Das \ac{SNIK} ist ein abgeschlossenes Projekt, welches Begriffe des Informationsmanagements in dem Teilgebiet der  medizinischen Informatik sowie deren Beziehungen untereinander beschreibt.
Es ist in der Lage, dieses Wissen in einer Ontologie darzustellen und diese sowohl maschinen- als auch menschenlesbar auszugeben.
Dabei nutzt es ein dem Semantic Web Stack ähnliches Modell (\cref{img:semanticwebstack2}), welches Anwendungen wie dem \ac{SNIK} Graph ermöglicht, Informationen abzufragen und diese Endnutzern in aufbereiteter Form zur Verfügung zu stellen.
\begin{figure}
\centering
\includegraphics[width=0.7\textwidth]{images/swebstackde_snik.pdf}
\caption{Das semantische Modell von SNIK.}
\label{img:semanticwebstack2}
\end{figure}
Das Projekt konzentriert sich hauptsächlich auf Wissen der medizinischen Lehre. Dieses wird für Krankenhausinformationssysteme zur Verfügung gestellt, um die Arbeit für medizinisches Personal zu erleichtern. Speziell hat der durch Prof. Dr. Alfred Winter geleitete Teilbereich \ac{MIG} die Zielsetzung, durch Entwicklung von Methoden und Werkzeugen für das Informationsmanagement zu einer besseren Gesundheitsversorgung beizutragen.

Um normalen Nutzern eine Anwendung, die die Daten des Projektes nutzt, darzustellen, wurde unter anderem auch ein einfaches Multiple-Choice-Quiz auf Basis eines öffentlich verfügbaren Templates programmiert.
Diese nutzt die Ontologie hinter SNIK, um Anwendern einfache, automatisch generierte Fragen zu stellen.
Diese Fragen werden mithilfe der typisierten, gerichteten Verbindungen zwischen einzelnen Objekten erzeugt, wobei das Quiz aus der SNIK-Ontologie drei Haupttypen von Verbindungen nutzt:
\todo{bessere Formatierung finden}
\todo{generell schachtelsätze in mehrere kleine sätze aufteilen}
\todo{anderes wort für normal, identifizieren, was diese auszeichnet}
\begin{itemize}
    \item 1. Objekt A definiert Objekt B (A defines B),
    \item 2. Objekt A beinhaltet Objekt B (A contains B) und
    \item 3. Objekt A unterstützt Objekt B (A supports B).
\end{itemize}
Dadurch werden für jeden Verbindungstyp unterschiedliche Fragen generiert.
Für Definitionen generiert sich der Fragetitel hierbei zum Beispiel nach dem folgenden Muster:
\newline \enquote{What is defined by A?}\newline
%TODO: mehr auf Antwortmöglichkeiten eingehen ( -> wie werden die generiert? )

\todo{
\begin{itemize}
\item In welcher Welt/Domäne oder welchem Arbeitsbereich/-gebiet bewegen wir uns im Rahmen der Seminararbeit/der ausgewählten Papers?
\item Worum geht es eigentlich?
\end{itemize}
Aus den Papers bzw. dem Antrag des Forschungsprojektes entnehmen.
}
%meiner Meinung nach ist das erledigt (TP) 

\subsection{Problematik}
\todo{generell im gesamten text: jeden satz auf eine zeile}
Die Nutzer sollen möglichst mit logischen, aber nicht zu einfachen Fragen herausgefordert werden.
Um mithilfe des Multiple-Choice-Quiz ihren Wissensstand zu verbessern, sollte das Quizprogramm bestimmte Vorgaben erfüllen und diese auch in der Qualität der Fragen widerspiegeln.
Das stellt sich als relativ schwer heraus, da die Implementierung von natürlichsprachlicher Grammatik sich seit jeher als ein Problem in der Informatik herausstellt.
\todo{komplexität und qualität: genauer definieren}
\todo{Sätze nicht mit doppelbackslash abschließen. latex kümmert sich selbst um den blocksatz.}
\paragraph{Korrektheit}
Für den Anwender soll die Frage gut lesbar und in seiner Grammatik verständlich sein.
Da die Quizfragen in einem bestimmten Zeitfenster bearbeitet werden sollen, ist es wichtig, dass sie schnell vom Nutzer verstanden werden.
Da der Hauptzweck die Selbstüberprüfung ist, werden dadurch fehlende Korrektheit gegenüber dem User die Ergebnisse verfälscht , was zu einer falschen Selbsteinschätzung führt.
\paragraph{Übersichtlichkeit}
Eine Frage muss für den Nutzer übersichtlich gestaltet sein.
Sie soll schnell durchgelesen sein, damit er den metaphorischen \enquote{Faden} nicht verliert und Fragen nicht erneut lesen muss.
Um dies zu garantieren, muss die Anzahl der Zeichen im Fragetitel eine bestimmte Zahl nicht übersteigen.
Dies kann zum Beispiel bei langen Definitionen (Fragetyp 1) ein Problem sein.
\paragraph{Qualität}
Neben den anderen beiden Bedingungen muss bei Fragen auch die Qualität und das Niveau der Frage sowie des Quiz allgemein hoch gehalten werden.
Doch das ist im bereits existenten Quiz an vielen Stellen nur teilweise vorhanden, was sich in verschiedenen Aspekten widerspiegelt.
Darunter fallen zum Beispiel die Erwähnung der richtigen Antwort im Fragetitel bei z.B. Definitionen oder in einfacheren Punken wie z.B. dem Doppeln von Fragen.
Dadurch wird der Nutzer nicht genug herausgefordert, was dazu führt, dass er die Fragen nicht mithilfe seines Fachwissens, sondern einfach durch logisches Denken beantworten kann. 
\paragraph{Irgendein Wort, welches beschreibt, dass etwas nicht andauernd gleich sein soll, mir fällt es sicher noch ein}
Das Quiz sollte für den Nutzer möglichst einen großen Teil der Möglichkeiten der SNIK-Ontologie anschaulich machen.
Da jedoch das existierende Quiz nur zwei der 18 Typen von Verbindungen zwischen Objekten nutzt, die in der SNIK-Ontologie vorkommen, variieren die Fragetypen nur wenig, was auf Dauer für Nutzer langweilig oder anstrengend wird. Außerdem wird 
\\\\\todo{
\begin{itemize}
\item Worin bestehen die Probleme?
\item Warum ist die geschilderte vorliegende Situation problematisch?
\item Für wen ist sie problematisch?
\end{itemize}
Aus den Papers bzw. dem Antrag des Forschungsprojektes entnehmen.\\
}\todo{
Hier ist auf die generelle Problemlage einzugehen, die dem Aufsatz im Wissenschaftsfeld zu Grunde liegt.\\
Bsp.: Daten aus der Krankenversorgung stehen aus rechtlichen und technischen Gründen nicht für die Forschung zur Verfügung, sodass klinische Forscher eigene Datenerhebungen planen müssen.
}
%erfüllt soweit, evtl noch bissl Formulierung (TP)
\subsection{Motivation}
%snik.eu nehmen
Das SNIK-Projekt hatte sich als Ziel gesetzt, ein theoretisch und empirisch begründetes, sowie erprobtes Semantisches Netz des Informationsmanagements im Krankenhaus (SNIK), das die Konzepte des Informationsmanagements beschreibt und verbindet, zu kreieren.
Um Medizinstudenten und anderen Mitarbeitern aus dem medizinischen Umfeld eine beispielhafte Nutzungsmöglichkeit für diese Ontologie zu bieten und ihnen einen einfachen Zugriff neben dem SNIK-Graph auf das Themengebiet zu ermöglichen, kann ein Multiple-Choice-Quiz einen einfachen Einblick auf die Einsatzmöglichkeiten von einer Ontologie aus Krankenhaussystemen bereitstellen und Interessenten aus nichtinformatischen Teilgebieten eine Vorstellung ermöglichen.

\todo{
\begin{itemize}
\item Warum lohnt es sich, die genannten Probleme zu lösen?
\item Wer wird welchen Nutzen von dieser Abschlussarbeit haben?
\item Warum sind die Papers wichtig?
\end{itemize}
Aus den Papers bzw. dem Antrag des Forschungsprojektes entnehmen.
}
\todo{
Worin liegt der zusätzliche Nutzen, die beiden Ansätze zu vergleichen? Worin liegt der Nutzen, den Ansatz des Papers auf die Problematik des Forschungsprojektes anzuwenden?\\
Bsp.: Gelänge es, Daten aus der Krankenversorgung vollständig zu anonymisieren, könnten sie für beliebige Forschungsvorhaben genutzt werden.
}

\section{Zielsetzung}
Das Ziel meiner besonderen Lernleistung ist es, ein Konzept für ein Multiple-Choice-Quizprogramm zu entwickeln, welches mit der SNIK-Ontologie arbeitet und auf Basis von verschiedenen Regeln und Qualitäten nutzerfreundliche Fragen generiert. 
\\Ziel Z1:  Konzept für SPARQL-Queries für verschiedene Beziehungstypen 
\\Zeil Z2:  Regeln und Umsetzungen für die Generierung von Fragen und deren Antworten
\\\todo{
Beschreiben Sie kurz die Zielsetzung ihrer Seminararbeit.
Beachten Sie dazu die Beschreibung des Betreuers zu Ihrem Seminarthema.\\
Beispiel:\\
Es soll herausgearbeitet werden, ob einer der in den Papers vorgestellten Lösungsansätze im aktuellen Forschungsprojekt angewendet werden kann: „Untersuchung der Eignung des Anonymisierungskonzept der statistischen Landesämter für medizinische Daten unter besonderer Betrachtung der Risiken des Zusatzwissens behandelnder Ärzte.“
Es soll eine Übersicht erarbeitet werden, die es ermöglicht, in einem konkreten Anwendungsfall den dort geeigneten Lösungsansatz auszuwählen.
}

\section{Vorgehensweise}
\todo{
Stellen Sie Ihre Methode bzw. Herangehensweise dar.
Was ist Ihre konkrete Aufgabenstellung im Rahmen der Seminararbeit? Das könnte z.B. sein:
\begin{itemize}
\item Vorstellen verschiedener Lösungsansätze, Vergleich und Bewertung der Lösungsansätze, Literaturrecherche nach weiteren vergleichbaren Papers.
\item Bewertung eines Lösungsansatzes hinsichtlich einer konkreten Forschungssituation.
Literaturrecherche nach weiteren vergleichbaren Papers.
\end{itemize}
}

\todo{
Beschreiben Sie Ihre Vorgehensweise in einzelnen Schritten und stellen Sie die Reihenfolge der Abarbeitung dar.
Auf diese Punkte sollten Sie in der Diskussion wieder eingehen.
}
\todo{
\begin{enumerate}
\item Schritt 
\item Schritt 
\end{enumerate}
}
\todo{
Vermeiden Sie Formulierungen wie \enquote{Es ist nicht bekannt, ob...} oder \enquote{Es existiert kein...}.
Solche Formulierungen kehren in der Regel einfach das bereits angedachte Lösungsmodell um und postulieren das Fehlen der angedachten Lösung einfach als Problem.
... \\
Schritt n\\
Das ist ähnlich, wie wenn es in der Werbung hieße \enquote{Wenn Sie das Problem haben, dass Ihnen Aspirin fehlt, dann kaufen Sie doch Aspirin.}
Sinnvoller ist diese Aussage: \enquote{Wenn Sie das Problem haben, dass Ihnen der Kopf weh tut, dann kaufen Sie doch Aspirin.}
Es ist also bei der Problembeschreibung erforderlich, sich in die Lage dessen zu versetzen, den man mit der angedachten Lösung beglücken möchte.
Sein Problem ist zu ermitteln und so zu formulieren, er/sie das Problem wiedererkennt und dadurch geneigt ist, sich für die Lösung des Problems zu interessieren.
} 
\\Aufgabe A1 zu Ziel Z1: Konzepte für die Abfragen von den drei verschiedenenen Beziehungstypen
\\Aufgabe A2 zu Ziel Z1: Konzepte zur Generierung der falschen Antworten  (\enquote{close matches})
\\Aufgabe A3 zu Ziel Z2: Regeln für Fragetitel \idea{Länge, Formatierung, abhängig vom Typ}
\\Aufgabe A4 zum Ziel Z2: Formatierung von Titel und Fragen \idea{Verhindern von Dopplungen, Absätze für Lesbarkeit}
\\Aufgabe A5 zum Ziel Z2: Generelle Regeln für das Multiple-Choice-Quiz \idea{keine doppelten Fragen, Statistiken für die Auswertung auf Basis von gesammelten Daten}

\todo{(2--3 Seiten)}
\chapter{Grundlagen}

\section{Semantic Web}
\label{sec:semanticweb}
Das Semantic Web (oder auch: semantisches Netz) stellt eine Methode dar, um maschinenlesbares Wissen über das \ac{WWW} in Form von HTML-Dokumenten zu verbreiten.
Anders als bei der klassischen Website aber wird hier nicht nur primär Wert auf die Lesbarkeit durch einen Menschen, sondern auch auf die Maschinenlesbarkeit Wert gelegt, indem die Seite nebst dem normalen HTML-Code auch Informationen, die durch Computer gelesen werden können, hinterlegt. Diese sind durch Technologien des semantischen Webs ( \acs{RDF}, \acs{SPARQL}, ... ) einles- und verarbeitbar.

\subsection{\acs{WWW}}
\label{sec:www}
Das \ac{WWW} bildet ein verknüpftes Kommunikationsmodell zwischen allen Ressourcen und Nutzern des Internets, die das \ac{HTTP} nutzen.
Es wurde vom Direktor der \ac{W3C} Tim Berners-Lee 1991 entwickelt und ermöglicht den Zugriff auf sowie den Datenaustausch mit \acs{HTML}-Dokumenten.
Es definiert also das, was die Allgemeinheit als \enquote{das Web} bezeichnet.


\subsection{Linked Data}
\label{sec:linkeddata}
Der Begriff \enquote{Linked Data} bezieht sich auf eine Reihe bewährter Methoden zum Veröffentlichen und Verbinden strukturierter Daten im Web.
Es verknüpft Dateien aus dem semantischen Web untereinander, sodass sie durch semantische Abfragen benutzbar werden.
Linked Data basiert auf standardisierten Modellen für den Datenaustausch im Web, besipielsweise \acs{HTML}, um auch wie beim semantischen Prinzip maschinenlesbare Daten zu generieren.
Zielsetzung von Linked Data ist es, das Internet zu einer globalen, computerverbarbeitbaren Datenbank weiterzuentwickeln und so den Zugriff für technische Endgeräte weiter zu vereinfachen.

\subsection{\acs{RDF}}
\label{sec:rdf}
\ac{RDF} ist ein weiteres Standardmodell für den Datenaustausch im Web, welches in den 90er Jahren des letzten Jahrtausends seine Anfänge findet.
Im Gegensatz zu \ac{HTML} und \ac{XML} ist es in der Lage, enthaltene Informationen zu kombinieren und weiter zu verarbeiten, anstatt diese nur korrekt anzuzeigen.
Daher ist RDF auch oft als grundlegendes Repräsentationsformat für die Entwicklung im Semantic Web angesehen.
Es erweitert die Verknüpfungsstruktur von diesem, um \ac{URIs} einzubinden.
Solche URIs werden im semantischen Netz dazu genutzt, Beziehungen zwischen Dingen beziehungsweise zwischen den beiden Enden der Verknüpfung herzustellen, um Dateien zu verknüpfen, verfügbar zu machen und für Anwendungen freizugeben.
Dadurch können Daten zusammengeführt werden, die sich eigentlich im Schema grundlegend unterscheiden, ohne dass alle Datenkonsumenten geändert werden müssen.
Diese Zusammenführung geschieht in einem Schichtenmodell: Aussagen werden als Tripel modelliert und bilden eine Aussage, die aus Subjekt, Prädikat und Objekt zusammengefügt werden kann.
Man kann also einen Tripel mit einem einfachen Satz vergleichen: als Beispiel hier einmal der Teilsatz \textit{Google produziert Prozessoren}.
Hier stellt \textit{Google} das Subjekt dar, \textit{produziert} ist das Prädikat und \textit{Prozessoren} sind das Objekt.
Hat man eine Menge von Tripeln, bilden diese ein semantisches Netz, was tabellarisch übersichtlich dargestellt werden kann, siehe \cref{tab:rdfexample}.

\begin{table}
\begin{centering}
\begin{tabularx}{\textwidth}{XXX}
\toprule
\textrm{Subjekt}			&\textrm{Prädikat}			&\textrm{Objekt}\\
\midrule
New York				&liegt in				&Amerika\\
Adobe				    &verkauft				&Software\\
Google					&produziert				&Prozessoren\\
\bottomrule
\end{tabularx}
\end{centering}
\caption{Beispiel für RDF-Tripel.\todo{prozessoren ist eher klasse als instanz}}
\label{tab:rdfexample}
\end{table}


\subsection{\acs{SPARQL}}
\ac{SPARQL} ist ein Standard für die Abfrage von in \ac{RDF} spezialisierten Informationen sowie für die Darstellung der zugehörigen Resultate, siehe \cref{img:semanticwebstack1}.
Es basiert auf einfachen \ac{RDF}-Anfragen in Form von Graphmustern, also einer Menge von Tripeln, enthält jedoch auch erweiterte Funktionen für die Konstruktion komplexerer Anfragemuster, die Verwendung zusätzlich hinzufügbarer Filterbedingungen und für die Formatierung der schlussendlichen Ausgabe.


Man kann also \ac{SPARQL} als eine faktische Abfragesprache des semantischen Webs bezeichnen.
\begin{figure}
\centering
\includegraphics[width=0.7\textwidth]{images/swebstackde.pdf}
\caption{Der Semantic Web Stack.}
\label{img:semanticwebstack1}

\end{figure}
\subsection {Ontologien}
\label{sec:ontologien}
\setlength\epigraphwidth{.9\textwidth}
\setlength\epigraphrule{0pt}

\epigraph{\itshape An ontology is an explicit specification of a conceptualization.}{---Thomas R. Gruber\todo{quelle zitieren}}

\noindent Ontologien sind (im informatischen Kontext) meist sprachlich gefasste und geordnete Darstellungen einer Menge von Begrifflichkeiten mit festen Beziehungen untereinander.
Diese Beziehungen befassen sich stets auf einen bestimmten Gegenstandsbereich und werden dazu genutzt, Wissen in digitalisierter Form zwischen Anwendungen und Diensten auszutauschen.
Dabei müssen bestimmte Interferenz- und Integritätsregeln eingehalten werden, also Regeln zu Schlussfolgerungen sowie zu der Gewährleistung ihrer Gültigkeit.
Ontologien erfreuen sich seit der Idee des semantischen Webs einem stets wachsenden Bekanntheitsgrad, was dazu führt, dass sie als Teil der Wissensrepräsentation im Teilgebiet \acs{KI} einen großen Einfluss haben.
Dabei beschreibt die Idee des Semantic Web eine Erweiterung des vorhandenen Netzes, um Daten zwischen Rechnern einfacher austausch- und verwertbar zu machen und sie somit besser zu explizieren, anstatt sie unkonstruiert stehen zu lassen.
Zum Sinne dieser Realisierung dienen Standards zur Veröffentlichung und Nutzung maschinenlesbarer Daten -- insbesondere \ac{RDF}.

\subsection{Künstliche Intelligenz}
\ac{KI} ist ein Teilgebiet der Informatik, welches sich mit der Automatisierung intelligenten Verhaltens befasst.
Es spiegelt eine Möglichkeit wieder, bestimmte Entscheidungsstrukturen des Menschen nachzubilden und es somit Computern zu ermöglichen, komplexe Probleme selbstständig zu bearbeiten und zu lösen.
Um diese Informationen weiter zu nutzen, können Methoden des Semantic Web genutzt werden, um sie als Tripel einfach maschinenlesbar darzustellen und für andere Computer verfügbar zu machen.

\section{Informationssysteme}
Ein Informationssystem ist ein System, welches durch die Bildung logischer Zusammenhänge eine Deckung der Informationsnachfrage zur Aufgabe hat.
Es produziert, beschafft, verteilt und verarbeitet Daten durch eine Zusammenarbeit zwischen Mensch und Technik durch Aufgaben.

\subsection{Krankenhausinformationssystem}
Krankenhausinformationssysteme beschreiben die Gesamtheit aller Informationssysteme zur Produktion, Beschaffung, Verteilung und Verarbeitung von medizinischen und administrativen Daten im Krankenhaus.
Dazu gehören verschiedene Formen der Datenbereitstellung sowie auch konventionelle Methoden der papierbasierten Dokumentation und der sprachlichen Kommunikation.
Das grundsätzliche Ziel eines Krankenhausinformationssystems ist, die Kommunikation zwischen Mitarbeitern zu verbessern und den Ablauf in Krankenhäusern zu steuern, indem Mitarbeitern gezielt Zugriff auf für ihn relevantes Wissen aus der ihm zugeteilten Benutzerrolle gegeben wird, zum Beispiel über den gerade zu behandelnden Patienten.
\subsection{Grundbegriffe}
\paragraph{Information}
\todo{Information definieren}
\paragraph{Wissen}
Wissen ist die Kenntnis über den in einem Fachgebiet zu gegebener Zeit gegebenen Konsens, vor allem bezogen auf eine gültige Terminologie, erlaubte Interpretationen, bestehender Zusammenhänge und Gesetzmäßigkeiten sowie empfehlenswerter Methoden und Handlungen.
Wissen ist also auch Information, aber in dem Fall nicht auf einzelne Objekte, sondern eine Menge von Objekten und deren Beziehungen untereinander bezogen, vgl. \citet{his}.
\paragraph{Daten}
Daten sind Gebilde aus Zeichen oder kontinuierliche Funktionen, die durch bekannte oder unterstellte Beziehungen Informationen darstellen können.
Sie stellen die Grundlage oder das Ergebnis eines Verarbeitungsschrittes dar. 

%Beispiel-Query:
\begin{comment}
\begin{lstlisting}
SELECT SAMPLE(replace(str(?def),str(?cl),"X","i") as ?def)
SAMPLE(str(?cl) as ?cl) 
SAMPLE(str(?a1l) as ?a1l)
SAMPLE(str(?a2l) as ?a2l)
SAMPLE(str(?a3l) as ?a3l)
{
 ?class a owl:Class.
 ?class rdfs:label ?cl.
 FILTER(LANGMATCHES(LANG(?cl),"en"))

 ?class skos:definition ?def.
 FILTER(STRLEN(?def)>10&&STRLEN(?def)<600).
 FILTER(LANGMATCHES(LANG(?def),"en"))

 ?class (!meta:subTopClass){1,2} ?a1,?a2,?a3.

 owl:Class ^a ?a1,?a2,?a3.
 FILTER(?class!=?a1&&?class!=?a2&&?class!=?a3
 &&?a1<?a2&&?a2<?a3)

?a1 rdfs:label ?a1l.
 FILTER(LANGMATCHES(LANG(?a1l),"en"))

?a2 rdfs:label ?a2l.
 FILTER(LANGMATCHES(LANG(?a2l),"en"))

?a3 rdfs:label ?a3l.
 FILTER(LANGMATCHES(LANG(?a3l),"en"))
 
} GROUP BY ?class limit 1000
\end{lstlisting}
\end{comment}

\section{\acs{SNIK}-Projekt}
Das \ac{SNIK} ist ein vollendetes Projekt, welches Begriffe des Informationsmanagements sowie deren Beziehungen untereinander beschreibt.
Es ist in der Lage, diese Menge an Informationen in einer Ontologie darzustellen und diese sowie maschinen- als auch menschenlesbar auszugeben.
 Dabei nutzt es ein dem Semantic Web Stack ähniches Modell(\cref{img:semanticwebstack2}), welches Anwendungen wie dem \ac{SNIK} Graph oder einem Multiple-Choice-Quiz ermöglicht, Informationen abzufragen und diese Endnutzern in aufbereiteter Form zur Verfügung zu stellen.
Dabei konzentriert sich das SNIK-Projekt hauptsächlich auf die Daten der medizinischen Lehre, um diese für Krankenhausinformationssysteme zur Verfügung zu stellen und die Arbeit für medizinisches Personal zu erleichtern.
\begin{figure}
\centering
\includegraphics[width=0.7\textwidth]{images/swebstackde_snik.pdf}
\caption{Das semantische Modell von SNIK.}
\label{img:semanticwebstack2}
\end{figure}

\chapter{Lösungsansatz/Lösungsansätze \todo{(2--4 Seiten)}}
\todo{Kapitel 3 ist Vorschlag gedacht, der nicht wörtlich übernommen werden muss.
Der Lösungsansatz ist eine kurze Beschreibung der Arbeitshypothese sowie des Vorgehens zur Lösung der in der Einleitung beschriebenen Probleme.}

\section{Ergebnis/Lösungsansatz Paper n}
\section{Vergleich mit Projekt}

\chapter{Ergebnis}
\todo{
Im Ergebniskapitel soll beschrieben werden, inwiefern Sie Ihre in 1.2/1.3 aufgestellten Ziele bzw. Aufgaben erreicht habe oder auch weshalb sie (teilweise) nicht erreicht werden konnten.
So soll es möglich sein, dass ein Leser von der Arbeit lediglich die Einleitung und die Zusammenfassung liest und doch die Ergebnisse der Arbeit erfassen kann.
Dieses Kapitel kann auch in mehrere Unterkapitel aufgeteilt werden, wenn das sinnvoll ist!
}

\chapter{Diskussion und Ausblick \todo{1--2 Seiten}}
\todo{
In der Diskussion wird das Ergebnis der Arbeit noch einmal kritisch bewertet.
Dies umfasst auch neue Probleme, die erst während der Bearbeitung erkannt wurden.
Die Eignung für ein konkretes Forschungsprojekt sollte hier diskutiert werden.
In der Diskussion kann auch die Kritik des Autors an dem stehen, was er in der Literatur zu dem zu bearbeitenden Thema hier und da gelesen hat.
Außerdem soll ein Ausblick gegeben werden, welche weiteren Fragestellungen noch bearbeitet werden sollten.
}

\renewcommand{\bibpreamble}{
\todo{
Zentrale Literatur, die den untersuchten Artikeln zugrunde liegt, können Sie im bibtex-Format in die seminar.bib einfügen und dann im Text zitieren, wodurch diese automatisch in das Literaturverzeichnis übernommen werden.
Sie können auch weitere referierte Veröffentlichungen (wissenschaftliche Zeitschriften (auch elektronisch), Bücher) in die Erarbeitung einbeziehen zitieren.
Sie können dafür Literaturdatenbanken wie Pubmed oder Google Scholar durchsuchen und dort direkt als bibtex-Eintrag herunterladen.
Bitte beachten Sie, dass die \href{https://www.openoffice.org/bibliographic/bibtex-defs.html}{bibtex-Einträge vollständig sind}, z.B. sind das bei Büchern Autor, Editor, Title, Kapitel oder Seiten, Herausgeber und Jahr..
Weitere Informationen zum Thema der Literaturrecherche finden sich in den \href{http://www.imise.uni-leipzig.de/Lehre/MedInf/Abschlussarbeiten/Literaturrecherche.jsp}{Hinweisen zur Literaturrecherche}.
\paragraph{Beispielzitierungen}
\citet{his} beschreiben ein Verfahren zur X von Y auf Basis von Z.
Alternativ: X von Y lässt sich auf Basis von Z ermitteln~\citep{his}.
}
}

\bibliographystyle{dinat}
\bibliography{bell}

\end{document}

